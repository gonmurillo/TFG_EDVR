\apendice{Plan de Proyecto Software}

\section{Introducción}

En este apartado se va a exponer la planificación temporal, la parte más importante de un proyecto. Indicando las tareas que se realizaron y cuándo se realizaron.  También se incluye un análisis de la viabilidad tanto económica como legal del proyecto.



\section{Planificación temporal}

La planificación temporal se ha realizado implementando una metodología ágil, el método \emph {Scrum}. 

\begin{itemize}

\item El desarrollo del proyecto se ha dividido en \emph {sprints}, con una duración de una semana o dos.

\item Los \emph {sprints} contienen las tareas que se llevaron a cabo en el intervalo de tiempo definido.

\item Cada tarea tiene asociada un coste aproximado basado en la dificultad o tiempo estimado de realización. 

\item En caso de que la estimación planificada fuese inexacta, ya sea mayor o menor, se modificó reflejando el tiempo real dedicado.

\item Al finalizar cada \emph {sprint} se realizaba una reunión junto a los tutores para explicar las dificultades encontradas, dudas y planificar los siguientes \emph {sprints}.

\end{itemize}

\subsection{Sprint 1}

	Fecha: 17/11/2020 - 1/12/2020
	
	El primer \emph {sprint} consistió en realizar una exploración bibliográfica inicial sobre el estado del arte e instalación en el equipo personal. 
	
	\begin{table}[H]
		 \begin{tabularx}{\linewidth}{X r r}
		 	\toprule \textbf{Tareas} & \textbf{Estimado} & \textbf{Final}\\
		 	\toprule
		 	Exploración bibliográfica inicial & 8 & 10 \\
		 	Instalación EDVR  & 6 & 4 \\
		 	\bottomrule
		 \end{tabularx}
		 \caption{Tareas del sprint 1}
	\end{table}

\subsection{Sprint 2}
    Fecha: 1/12/2020 - 15/12/2020

    Se continuó con la exploración bibliográfica centrándose en los documentos de EDVR. También se estudiaron las posibilidades de instalación ante la incompatibilidad con mi equipo.


    \begin{table}[H]
    	 \begin{tabularx}{\linewidth}{X r r}
    	 	\toprule \textbf{Tareas} & \textbf{Estimado} & \textbf{Final}\\
    	 	\toprule
            Continuación del estudio & 10 & 10 \\
         	Exploración de alternativas & 2 & 2 \\
            Instalación de VPN & 1 & 1 \\
    	 	\bottomrule
    	 \end{tabularx}
    	 \caption{Tareas del sprint 2}
    \end{table}
    
\subsection{Sprint 3}

    Fecha: 15/12/2020 - 29/12/2020
    
    Se comenzó la instalación en el equipo remoto y también la descarga de los conjuntos de datos REDS y Vimeo 90K.
    
    \begin{table}[H]
    	 \begin{tabularx}{\linewidth}{X r r}
    	 	\toprule \textbf{Tareas} & \textbf{Estimado} & \textbf{Final}\\
    	 	\toprule
            Instalación en el equipo remoto & 8 & 8 \\
         	Comienzo de descarga de los conjuntos de datos  & 5 & 13 \\
    	 	\bottomrule
    	 \end{tabularx}
    	 \caption{Tareas del sprint 3}
    \end{table}

\subsection{Sprint 4}

    Fecha: 29/12/2020 - 12/1/2021
    
    La instalación del conjunto de datos REDS dio más problemas de los esperados y se continuo con su descarga. También con Vimeo 90K descargado, se empezaron a hacer pruebas de ejecución con ese conjunto de datos.
    
    \begin{table}[H]
    	 \begin{tabularx}{\linewidth}{X r r}
    	 	\toprule \textbf{Tareas} & \textbf{Estimado} & \textbf{Final}\\
    	 	\toprule
            Continuación de descarga de REDS & 10 & 10 \\
         	Primeras ejecuciones  & 4 & 6 \\
    	 	\bottomrule
    	 \end{tabularx}
    	 \caption{Tareas del sprint 4}
    \end{table}

\subsection{Sprint 5}

    Fecha: 12/1/2021 - 27/1/2021
    Se continuo con las ejecuciones con Vimeo 90K y se empieza con REDS.
    
    \begin{table}[H]
    	 \begin{tabularx}{\linewidth}{X r r}
    	 	\toprule \textbf{Tareas} & \textbf{Estimado} & \textbf{Final}\\
    	 	\toprule
         	Continuación con Vimeo 90K  & 4 & 4 \\
         	Comienza la ejecución con REDS  & 4 & 4 \\
    	 	\bottomrule
    	 \end{tabularx}
    	 \caption{Tareas del sprint 5}
    \end{table}

\subsection{Sprint 6}
    Fecha: 27/1/2021 - 3/2/2021
    Se continuo con la ejecución de REDS.
    \begin{table}[H]
    	 \begin{tabularx}{\linewidth}{X r r}
    	 	\toprule \textbf{Tareas} & \textbf{Estimado} & \textbf{Final}\\
    	 	\toprule
         	Continuación con REDS  & 3 & 4 \\
    	 	\bottomrule
    	 \end{tabularx}
    	 \caption{Tareas del sprint 6}
    \end{table}

\subsection{Sprint 7}

    Fecha: 3/2/2021 - 17/2/2021
    
    Se comenzó a trabajar en un cuaderno de Jupyter, con la instalación de EDVR y explicaciones para entenderlo y ejecutarlo. También se creó un repositorio de GitHub.
    
    \begin{table}[H]
    	 \begin{tabularx}{\linewidth}{X r r}
    	 	\toprule \textbf{Tareas} & \textbf{Estimado} & \textbf{Final}\\
    	 	\toprule
         	Comienzo cuaderno de demostración Jupyter  & 5 & 5 \\
        	Creación del repositorio en GitHub & 2 & 4 \\
    	 	\bottomrule
    	 \end{tabularx}
    	 \caption{Tareas del sprint 7}
    \end{table}

\subsection{Sprint 8}

    Fecha: 17/2/2021 - 24/2/2021
    
    Se continuó con el cuaderno de Jupyter, con más explicaciones y la ejecución. El repositorio de GitHub, se actualizó con la mayoría del trabajo realizado hasta la fecha.
    
    \begin{table}[H]
    	 \begin{tabularx}{\linewidth}{X r r}
    	 	\toprule \textbf{Tareas} & \textbf{Estimado} & \textbf{Final}\\
    	 	\toprule
         	Continuación con Jupyter  & 5 & 5 \\
	        Creación del repositorio en GitHub & 3 & 4 \\
    	 	\bottomrule
    	 \end{tabularx}
    	 \caption{Tareas del sprint 8}
    \end{table}

\subsection{Sprint 9}
    
    Fecha: 24/2/2021 - 10/3/2021
    
    Se continuó con el cuaderno de Jupyter, con la incorporación del mini conjunto de pruebas. Se empezó a estudiar 
    como implementar un video propio para ser procesado.
    
    \begin{table}[H]
    	 \begin{tabularx}{\linewidth}{X r r}
    	 	\toprule \textbf{Tareas} & \textbf{Estimado} & \textbf{Final}\\
    	 	\toprule
         	Continuación con Jupyter  & 6 & 5 \\
        	Comienzo video propio & 6 & 6 \\
    	 	\bottomrule
    	 \end{tabularx}
    	 \caption{Tareas del sprint 9}
    \end{table}
    
    \subsection{Sprint 10}
    
    Fecha: 10/3/2021 - 24/3/2021
        
    Se empezaron las pruebas con el video propio. Se añadieron los últimos retoques al cuaderno Jupyter.
    
    \begin{table}[H]
    	 \begin{tabularx}{\linewidth}{X r r}
    	 	\toprule \textbf{Tareas} & \textbf{Estimado} & \textbf{Final}\\
    	 	\toprule
         	Pruebas con el video propio  & 8 & 12 \\
	        Añadidos finales cuaderno & 4 & 2 \\
    	 	\bottomrule
    	 \end{tabularx}
    	 \caption{Tareas del sprint 10}
    \end{table}

\subsection{Sprint 11}

    Fecha: 24/3/2021 - 31/3/2021
    
    Se continuó con la ejecución del video propio, consiguiendo obtener los resultados deseados.
    
    \begin{table}[H]
    	 \begin{tabularx}{\linewidth}{X r r}
    	 	\toprule \textbf{Tareas} & \textbf{Estimado} & \textbf{Final}\\
    	 	\toprule
         	Pruebas con el video propio  & 8 & 8 \\
    	 	\bottomrule
    	 \end{tabularx}
    	 \caption{Tareas del sprint 11}
    \end{table}

\subsection{Sprint 12}

    Fecha: 31/3/2021 - 14/4/2021

    Se estudió la posibilidad de realizar el entrenamiento con videos propios y diferentes técnicas de métricas de clasificaciones de imágenes.
  
    \begin{table}[H]
    	 \begin{tabularx}{\linewidth}{X r r}
    	 	\toprule \textbf{Tareas} & \textbf{Estimado} & \textbf{Final}\\
    	 	\toprule
     		Train  & 3 & 8 \\
    		Métricas de clasificación & 6 & 6\\
    	 	\bottomrule
    	 \end{tabularx}
    	 \caption{Tareas del sprint 12}
    \end{table}

\subsection{Sprint 13}

    Fecha: 14/4/2021 - 21/4/2021
    
    Se adquirieron más vídeos con distinto contenido para ser procesados. Se busca un método para convertir las imágenes procesadas a vídeo.
    
    \begin{table}[H]
    	 \begin{tabularx}{\linewidth}{X r r}
    	 	\toprule \textbf{Tareas} & \textbf{Estimado} & \textbf{Final}\\
    	 	\toprule
         	Más vídeos   & 5 & 5 \\
		    Fotograma a vídeo & 4 & 4\\
    	 	\bottomrule
    	 \end{tabularx}
    	 \caption{Tareas del sprint 13}
    \end{table}

\subsection{Sprint 14}

    Fecha: 21/4/2021 - 28/4/2021
    
    Se empezó a explorar el tema de la interfaz, valorando distintas opciones y decidiéndose por PySimpleGUI. Estudio de la biblioteca.
    
    \begin{table}[H]
    	 \begin{tabularx}{\linewidth}{X r r}
    	 	\toprule \textbf{Tareas} & \textbf{Estimado} & \textbf{Final}\\
    	 	\toprule
         	Opciones de interfaz   & 6 & 4 \\
		    Estudio PySimpleGUI & 8 & 3\\
    	 	\bottomrule
    	 \end{tabularx}
    	 \caption{Tareas del sprint 14}
    \end{table}
    
\subsection{Sprint 15}

    Fecha: 28/4/2021 - 5/5/2021
    
    Se comenzó a desarrollar la interfaz, convertir el vídeo en fotogramas, usando Matlab para transformar la imagen y fotogramas a vídeo.
    
    \begin{table}[H]
    	 \begin{tabularx}{\linewidth}{X r r}
    	 	\toprule \textbf{Tareas} & \textbf{Estimado} & \textbf{Final}\\
    	 	\toprule
         	Opciones de interfaz   & 6 & 4 \\
		    Vídeo a fotogramas  & 2 & 2\\
		    Matlab & 4 & 7 \\
	    	Fotogramas a vídeo & 3 & 3 \\
    	 	\bottomrule
    	 \end{tabularx}
    	 \caption{Tareas del sprint 15}
    \end{table}

\subsection{Sprint 16}

    Fecha: 5/5/2021 - 12/5/2021
    
    Se busca e implementa una alternativa a Matlab. Empieza el estudio sobre cómo usar \LaTeX
    
    \begin{table}[H]
    	 \begin{tabularx}{\linewidth}{X r r}
    	 	\toprule \textbf{Tareas} & \textbf{Estimado} & \textbf{Final}\\
    	 	\toprule
		    Sustituto Matlab & 4 & 7 \\
		    Estudio \LaTeX & 3 & 3 \\
    	 	\bottomrule
    	 \end{tabularx}
    	 \caption{Tareas del sprint 16}
    \end{table}

\subsection{Sprint 17}

    Fecha: 12/5/2021 - 19/5/2021
    
    Se comienza a escribir la memoria del trabajo de fin de grado.
    
    \begin{table}[H]
    	 \begin{tabularx}{\linewidth}{X r r}
    	 	\toprule \textbf{Tareas} & \textbf{Estimado} & \textbf{Final}\\
    	 	\toprule
         	Escribir la memoria   & 6 & 4 \\
    	 	\bottomrule
    	 \end{tabularx}
    	 \caption{Tareas del sprint 17}
    \end{table}

\subsection{Sprint 18}

    Fecha: 19/5/2021 - 9/6/2021
    
    Se continúa  escribiendo  la memoria del trabajo de fin de grado. Y se empieza con los anexos del trabajo de fin de grado.
    
    \begin{table}[H]
    	 \begin{tabularx}{\linewidth}{X r r}
    	 	\toprule \textbf{Tareas} & \textbf{Estimado} & \textbf{Final}\\
    	 	\toprule
         	Continuar escribiendo la memoria   & 20 & 20 \\
		    Empezar a escribir los anexos & 9 & 8 \\
    	 	\bottomrule
    	 \end{tabularx}
    	 \caption{Tareas del sprint 18}
    \end{table}

\subsection{Sprint 19}

    Fecha: 9/6/2021 - 19/6/2021
    
    Se continúa  escribiendo  los anexos del trabajo de fin de grado. 
    \begin{table}[H]
    	 \begin{tabularx}{\linewidth}{X r r}
    	 	\toprule \textbf{Tareas} & \textbf{Estimado} & \textbf{Final}\\
    	 	\toprule
         	 Continuar escribiendo los anexos & 19 & 15 \\
		
    	 	\bottomrule
    	 \end{tabularx}
    	 \caption{Tareas del sprint 19}
    \end{table}

\subsection{Sprint 20}

    Fecha: 19/6/2021 - 30/6/2021
    
    Se acaba de escribir los anexos del trabajo de fin de grado. Se realizan un para de figuras para la memoria. Y se retoca el contenido de Git Hub actualizandolo.
    \begin{table}[H]
    	 \begin{tabularx}{\linewidth}{X r r}
    	 	\toprule \textbf{Tareas} & \textbf{Estimado} & \textbf{Final}\\
    	 	\toprule
         	Fin los anexos & 10 & 12  \\
         	Figuras  & 8 & 8  \\
            Git hub & 5 & 5  \\
		
    	 	\bottomrule
    	 \end{tabularx}
    	 \caption{Tareas del sprint 19}
    \end{table}

\subsection{Coste total de los \emph{Sprints}}

En la tabla~\ref{tab:costes_sprints} se muestran los costes estimados y finales totales de cada sprint y la suma del coste final del proyecto. 
	

	\begin{table}[H]
		\begin{tabularx}{\textwidth}{Xrr}
			\toprule \textbf{\textit{Sprint}} & \textbf{Estimado} & \textbf{Final}\\
			\toprule
			Sprint 1 & 14 & 14 \\
			Sprint 2 & 13 & 13 \\
			Sprint 3 & 13 & 21 \\
			Sprint 4 & 14 & 16 \\
			Sprint 5 & 8 & 8 \\
			Sprint 6 & 3 & 4 \\
			Sprint 7 & 7 & 9 \\
			Sprint 8 & 8 & 9 \\
			Sprint 9 & 12 & 11 \\
			Sprint 10 & 12 & 14 \\
			Sprint 11 & 8 & 8 \\
			Sprint 12 & 9 & 14 \\
			Sprint 13 & 9 & 9 \\
			Sprint 14 & 14 & 7 \\
			Sprint 15 & 15 & 16 \\
			Sprint 16 & 7 & 10 \\
			Sprint 17 & 6 & 4 \\
			Sprint 18 & 29 & 28 \\
			Sprint 19 & 19 & 15 \\
	        Sprint 20 & 23 & 25 \\
			\midrule
			\textbf{Total} & 243 & 245 \\
			\bottomrule
		\end{tabularx}
		\caption{Coste de cada sprint.}
		\label{tab:costes_sprints}
	\end{table}


\section{Estudio de viabilidad}

\subsection{Viabilidad económica}
Para considerar la viabilidad económica del proyecto se deben calcular
los costes derivados de su realización. Se van a tener en cuenta tanto el
coste del personal como el del software y el hardware empleados.
\subsection{Costes de personal}
Siguiendo lo expuesto anteriormente el coste personal es:
\begin{table}[H]
		\centering
		\begin{tabular}[]{@{}l r@{}}
			\toprule
			\textbf{Concepto} & \textbf{Coste(\euro{})} \\
			\midrule
			Salario mensual neto & 968,63 \\
			Retención IRPF (15\%) & 256,24 \\
			Seguridad Social (28,3\%) & 483,46 \\
			Salario mensual bruto & 1\,708,33 \\
			\midrule
			\textbf{Total 7 meses} &  11\,958,31 \\
			\bottomrule
		\end{tabular}
		\caption{Costes de trabajador.}
		\label{tab:costes_trabajadorl}
	\end{table}

\subsection{Costes de software}

El único \emph{software} de este trabajo que es de pago es Matlab para estudiantes, en mi caso la Universidad de Burgos nos provee con licencia. En este supuesto se incluirá el precio que se debería pagar por una licencia básica. 
\begin{table}[H]
		\centering
		\begin{tabular}[]{@{}l r@{}}
			\toprule
			\textbf{Concepto} & \textbf{Coste(\euro{})} \\
			\midrule
			Licencia Matlab students & 69\\
			\midrule
			\textbf{Total } &  69,00 \\
			\bottomrule
		\end{tabular}
		\caption{Costes software.}
		\label{tab:costes_software}
	\end{table}


\subsection{Costes de hardware}
Para el desarrollo de el trabajo no se ha adquirido ningún \emph{hardware}, por lo que se incluye los costos del material con el que ya se contaba, asumiendo una amortización en 5 años y calculando el costo de la amortización en la duración del trabajo, 7 meses.
\begin{table}[H]
		\centering
		\begin{tabular}[]{@{}l c r@{}}
			\toprule
			\textbf{Concepto} & \textbf{Coste(\euro{})} & \textbf{Coste amortizado(\euro{})} \\
			\midrule
			Dispositivo móvil & 150 & 17,5 \\
			Ordenador portátil  & 800 & 93,33 \\
			\midrule
			\textbf{Total} & 950 & 110,83 \\
			\bottomrule
		\end{tabular}
		\caption{Costes de \textit{hardware}.}
		\label{tab:costes_hardware}
	\end{table}

\subsubsection{Coste total}
	
	Teniendo en cuenta los costes de personal, de \emph{software} y de \textit{hardware}, el coste económico total del proyecto asciende a:
	

	\begin{table}[H]
		\centering
		\begin{tabular}[]{@{}l r@{}}
			\toprule
			\textbf{Concepto} & \textbf{Coste(\euro{})} \\
			\midrule
			Coste de personal & 11\,958,31 \\ 
            Coste de \textit{software} & 69,00\\
			Coste del \textit{hardware} & 950,00 \\
			\midrule
			\textbf{Total} & \textbf{12\,977,31} \\	
			\bottomrule	
		\end{tabular}
		\caption{Coste total.}
		\label{tab:coste_total}
	\end{table}
	
\subsection{Viabilidad legal}

Para el estudio de la viabilidad del producto, se van a analizar las bibliotecas 
usadas en el proyecto, anotando las licencias de las que hacen uso. A continuación se listan:

\begin{itemize}
\item \textbf {Zero clause BSD}: Tmux.
\item \textbf {Modified BSD}: Anaconda, Jupyter Notebook, Spyder, PyTorch.
\item \textbf {Apache 2.0}: OpenCv.
\item \textbf {GPLv3}: PySimpleGUI.
\item \textbf {LGPL}: ffmpeg.
\item \textbf {Comercial}: Matlab.
\end{itemize}
	




