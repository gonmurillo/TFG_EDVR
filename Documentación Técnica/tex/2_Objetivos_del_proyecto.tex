\capitulo{2}{Objetivos del proyecto}

En este apartado se exponen los objetivos perseguidos con la realización de este trabajo: 


\section{Objetivos generales}

\begin{itemize}
    \item Conseguir la instalación y funcionamiento de EDVR en el equipo de pruebas.
    
    \item Probar a implementar mis propios vídeos en la ejecución de EDVR y comparar resultados.
    
     \item Analizar el impacto de la mejora de estos vídeos sobre casos concretos en los que aparecen gestos y acciones humanas simples.
    
    \item Reducir el tiempo de descarga de datasets y modelos preentrenados necesarios para probar EDVR, para ello se  proporciona un mini conjunto de pruebas.
    
     \item Probar el funcionamiento de un algoritmo de aprendizaje automático para resolver problemas.
     
    \item Investigar sobre distintas técnicas de super resolución de vídeos e imágenes.

    \item Simplificación de la ejecución mediante una interfaz.
    
    \item Comprobar la viabilidad de su uso para vídeos en los que aparecen acciones humanas sencillas como gestos, así como algunas más complejas.
\end{itemize}


\section{Objetivos técnicos}

\begin{itemize}
    \item Aprender  \LaTeX{}, en concreto su uso en herramientas de edición online (Overleaf).
    
    \item Realizar y exponer los resultados de la demostración en un  notebook de Jupyter empleando código Python.
    
    \item Usar la plataforma GitHub para la gestión del proyecto.
    
    \item Aplicar la metodología Scrum en un proyecto real.
    
    \item Profundizar en el funcionamiento de la biblioteca Pytorch y sus capacidades de aprendizaje automático.
    
    \item EDVR ofrece la posibilidad de trabajar con Slurm para el computo distribuido, aprender cómo trabajar con él y sus posibilidades.
    
    \item Aprender a usar PySimpleGUI para dotar a los programas en Python de una interfaz básica pero funcional, con muchas opciones de personalización.
    
    
    
\end{itemize}


\section{Objetivos personales}

\begin{itemize}
    \item Iniciarme en el campo de la investigación.
    \item Explorar técnicas y herramientas del aprendizaje automático y la mejora de vídeos e imágenes.
    \item Familiarizarme con una herramienta como Tmux que facilita el trabajo multi tarea.
    \item Aprender a generar documentación con \LaTeX{}.
    \item Primera puesta en contacto con un desarrollo real.
    
\end{itemize}
